\begin{enumerate}
    
    \textbf{Intro into Irrational Numbers }\\
    An irrational number is a number that cannot be expressed as a fraction p/q for any integers p and q. Irrational numbers have decimal expansions that neither terminate nor become periodic.\hfill\linebreak
    \linebreak

    \textbf{Universal parabolic constant}\\
    
     The Universal Parabolic Constant  is a type of irrational number , its defined as the ratio, for any parabola or Semi circle, of the arc length of the parabolic segment formed by the latus rectum to the focal parameter Two time the focal length \hfill\linebreak
     
    \linebreak
    
    \begin{equation}
    {P=\ln(1+{\sqrt {2}})+{\sqrt {2}}=2.29558714939\dots }
    \end{equation}
 
 \textbf{Derivation}
\\
 P is described to be a  "Transcendental number", which when elaborated upon are real/complex numbers that are not a root of a nonzero polynomial equation. best examples of transcendental number are π and e.
 \\


    \begin{equation}
    
    {\displaystyle {P&:={\frac {1}{p}}\int _{-\ell }^{\ell }{\sqrt {1+\left({\frac {dy}{dx}}\right)^{2}}}
    \end{equation}
    
    \begin{equation}
    
    {dx&={\frac {1}{2f}}\int _{-2f}^{2f}{\sqrt {1+{\frac {x^{2}}{4f^{2}}}}}\,dx\\&=\int _{-1}^{1}{\sqrt {1+t^{2}}}\,dt\quad (x=2ft)\\&=\operatorname 

    \end{equation}
    
    \begin{equation}
    ={arcsinh} (1)+{\sqrt {2}}\\&=\ln(1+{\sqrt {2}})+{\sqrt {2}}}}
    \end{equation}


\\\\
 \textbf{Properties of the  Universal Parabolic Constant }
 
 \\
 
 P is described to be a  "Transcendental number", which when elaborated upon are real/complex numbers that are not a root of a nonzero polynomial equation. best examples of transcendental number are π and e. 
 
 \\
 \\
 \textbf{Applications of the Universal Parabolic Constant}
 \\
Universal Parabolic Constant can be used to calculate the average distance from a random point in the square unit to the center of the parabola. 

This can be used to measure practical distances during the constitution of real world objects such as Satellite Dishes where Incoming waves are concentrated on a singular focal point. Or architectural arcs where the extrema of arcs may need accurate points of measurements for their constructions 

 \begin{equation}
{d_{\text{avg}}={P \over 6}.}
 \end{equation}

 
 
\end{enumerate}