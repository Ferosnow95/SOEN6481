
\begin{enumerate}
\begin{flushleft}

Apart from a documentative exploration of my problem domain about Eternity Number, an exploratory interview was conducted for the creation of this report to understand the need of the Universal parabolic constant in practical environment and how laymen people come about to actually use the mathematical constant on a day to day basis.   

\textbf{Interview Structure}

To enable conformity of documentative process, the interview was structured into 4 Parts, with the parts labeled as follows 

\begin{quote}
\item{1.Introduction}
\item{2.Domain Concesces}
\item{3.Applications of the Constant}
\item{4.Introduction}
\hfill\break\\


\end{quote}
\textbf{Subject Background}

The Interview's background and biological information are explicitly detailed the persona documentation, The Interviewee subject in question for the project is \textbf{Haneen Zafar}, an undergraduate architecture Student at King Fahad University in mainland Saudi Arabia, \\
\hfill \break
\textbf{Interview Environment}

The interview was conducted in an isolated room environment over a phone call, to allow the subject to be calm and free to answer during the interviewing process, this was done to eliminate any sources of hesitation or social anxiety that may occur during the interviewing . The Interview was Started around 10:30 AM in the morning EST and lasted around 15 Mins excluding the conclusion .
\hfill \break

\textbf{Interview Structure}


\textbf{1. Introduction}

To make the interviewee feel relaxed and welcomed an initial set of questions focused understanding the hierarchical persona of the person  \hfill \break

   \begin{quote}
       
   
    \textbf{1. Please State your name and vocation for the purposes of the  interview ?}
    
    Answer : Hi Ali, My name is Haneen, I am an Architectural student currently in my second year in King Fahad University in Saudi Arabia.  
    \hfill \break
   
    \textbf{2. Can you elaborate on your field of study in University?}
    
    Answer : I Study mostly architectural works and construct modal concepts both physically and in software systems to simulate possible structures with real world constructs realized.
    \hfill \break 
    
    \textbf{3. Your work in architecture was one of the reasons I contacted you for this interview, I wanted to inquire , how often do you use Mathematical Constants in your studies/work? }
    
    Answer : I Study mostly architectural works and construct modal concepts both physically and in software systems to simulate possible structures with real world constructs realized.
    \hfill \break
    \end{quote} 
    
    
    
    \textbf{2. Gathering Opinion (Understanding the Number/Domain)}

    To make the interviewee feel relaxed and welcomed an initial set of questions focused understanding the hierarchical persona of the person  \hfill \break

    \begin{quote}
    \textbf{4. For my university project, I am supposed to figure out certain applications for a mathematical constant, are you familiar with the Universal Parabolic constant?}
    
    Answer : Yes actually It was thought to us in one of our classes here at the university, \hfill \break
   \linebreak
   
    \textbf{5. Oh Nice, So In your opinion What uses do architects find with this constant? }
    
    Answer : Haha, you'd be surprised, honestly I mean as the name states, Parabolas are used everywhere in Architecture, from Suspension bridges , Fountains  to modern art structures. The constant allows us to proximate distances of possible structural degradation/changes over time, this can be because of let's say Gravity or  differences caused by changes in environmental temperatures. \hfill \break
    \end {quote}
    
    \textbf{3. Applications of the Number}

    The section of the Interview focused on getting information of the use of the UPC. \hfill \break

     \begin {quote}
    \textbf{6. How often do you use or see yourself using the Universal Parabolic Constant?}
    
    Answer : Like I said it depends, mostly for our approximations we pre simulate our structural diagrams in blender or 3D Max for presentation purposes, however like the example I gave you, when you are calculating stuff like structural changes caused by heat you need to use approximation constants like the UPC combined with various other functions to calculate the differences in distances from median before and after the change. \hfill \break
   \linebreak
   
    \textbf{7. So Imagine, If an application was developed, what forms of functions would you like to perform with the constant?}
    
    Answer : Answer : Hmm, It would be nice if I could combine my pre existing functions to your Parabolic constant, Like we do on an actual calculator, but on the top of my head, I can't think of any heh. \hfill \break
     \end {quote}
    \textbf{4. Conclusion of the interview }

   The section of the Interview focused on debriefing the subject, and getting their consent on using the information obtained during the interviewing process for the purposes of this project. \hfill \break

    \begin {quote}
    \textbf{8. Thanks for your time Haneen, You can't Imagine how helpful your insights are to my project.}
    
    Answer : Oh, it was no problem,just the time difference was a bit of trouble on my end, as you can imagine I have classes \hfill \break
   \linebreak
   
    \textbf{9. Yes, Apologies for that, I have a short work notice for my project, So time is something that is fleeting for me. Can I contact you again If I am in need of further insights for my project? }
    
    Answer : Sure, Just be sure to let me know beforehand, as It would allow me to adjust my schedule for your call.\hfill \break
    \end {quote}\hfill\break
    
\textbf{Interview analysis}

The interview with Haneen allowed me to understand that the Parabolic constant is widely used outside the confines of the mathematics based research fields, which was my initial take as the primary use case domain for the constant number. 
    
    
    

\end{flushleft}





\end{enumerate}

