% Use only LaTeX2e, calling the article.cls class and 12-point type.

\documentclass[12pt]{report}
\usepackage[a4paper]{geometry}
\usepackage[myheadings]{fullpage}
\usepackage{fancyhdr}
\usepackage{lastpage}
\usepackage{graphicx, wrapfig, subcaption, setspace, booktabs}
\usepackage[T1]{fontenc}
\usepackage[font=small, labelfont=bf]{caption}
\usepackage{fourier}
\usepackage[protrusion=true, expansion=true]{microtype}
\usepackage[english]{babel}
\usepackage{sectsty}
\usepackage{url, lipsum}
\usepackage{tgbonum}
\usepackage{hyperref}
\usepackage{xcolor}
\usepackage{dingbat}

\pagestyle{fancy}
\setlength\headheight{15pt}
\fancyhead[L]{Student ID: 40076898}
\fancyhead[R]{Concordia University}


% Users of the {thebibliography} environment or BibTeX should use the
% scicite.sty package, downloadable from *Science* at
% www.sciencemag.org/about/authors/prep/TeX_help/ .
% This package should properly format in-text
% reference calls and reference-list numbers.



% Use times if you have the font installed; otherwise, comment out the
% following line.

\usepackage{times}

% The preamble here sets up a lot of new/revised commands and
% environments.  It's annoying, but please do *not* try to strip these
% out into a separate .sty file (which could lead to the loss of some
% information when we convert the file to other formats).  Instead, keep
% them in the preamble of your main LaTeX source file.


% The following parameters seem to provide a reasonable page setup.

\topmargin 0.0cm
\oddsidemargin 0.1cm
\textwidth 16cm 
\textheight 22cm
\footskip 1.0cm


%The next command sets up an environment for the abstract to your paper.

\newenvironment{sciabstract}{%
\begin{quote} \bf}
{\end{quote}}




% If your reference list includes text notes as well as references,
% include the following line; otherwise, comment it out.

\renewcommand\refname{References and Notes}

% The following lines set up an environment for the last note in the
% reference list, which commonly includes acknowledgments of funding,
% help, etc.  It's intended for users of BibTeX or the {thebibliography}
% environment.  Users who are hand-coding their references at the end
% using a list environment such as {enumerate} can simply add another
% item at the end, and it will be numbered automatically.

\newcounter{lastnote}
\newenvironment{scilastnote}{%
\setcounter{lastnote}{\value{enumiv}}%
\addtocounter{lastnote}{+1}%
\begin{list}%
{\arabic{lastnote}.}
{\setlength{\leftmargin}{.22in}}
{\setlength{\labelsep}{.5em}}}
{\end{list}}



 


% Include your paper's title here

\title{The Universal Parabolic Constant\\ \textbf{Eternity Numbers}}
\date{2018\\ July}
\author{By Ali Zafar Iqbal\\\\ Department of Computer Science \\Concordia University }

% Include the date 
\date{}
\begin{document} 
\maketitle 

% Double-space the manuscript.

\baselineskip20pt

% Make the title.

% Place your abstract within the special {sciabstract} environment.
\tableofcontents
\graphicspath{ {./Universal Parabolic Constant Mindmap/}{./images2/} }



\newpage




 


\chapter{User Stories}
    
   Project based user stories are created to try and comprehend the scope of the project's  base user requirements .  As one of the prerequisites for the project report,  The following initial User stories were created so characterize the user requirements for the calculator and its overarching functionalities 

\begin{quote}
                \section{US1 - Select and Input Calculation Numbers}
                
                \begin{tabular}{ |p{4cm}|p{10cm}| }
                 \hline
                 \multicolumn{2}{|c|}{\textbf{US1 - Select and Input Calculation Numbers}} \\
                 \hline
                 \textbf {Story ID}& US1  \\
                 \hline
                 \textbf{Priority} & HIGH \\
                 \hline
                 \textbf{Description}   & As an user of calculator, I want to easily input the digital numbers when I want an calculation to be done  \\
                 \hline
                 \textbf{Acceptance}& 
                
                 - I know I'm done when I press the displayed number is equal to the numanical press on the keyboard
                
                \\
                 \hline
                 \textbf{Estimate} &  0.4  point  \\
                 \hline
                 \textbf{Constrains}& The display cannot start from 0 ie Input 01 = "1" / Must not accept characters  \\
                 \hline
                \end{tabular}
            \hfill\break\\\\
                
        
        
        
        
        \section{US2 - Select Operands for Calculations}
                \begin{tabular}{ |p{4cm}|p{10cm}| }
                 \hline
                 \multicolumn{2}{|c|}{\textbf{US2 - Select Operands for Calculations}} \\
                 \hline
                 \textbf {Story ID}& US2  \\
                 \hline
                 \textbf{Priority} & HIGH \\
                 \hline
                 \textbf{Description}   &  As an user of calculator, I want to be able to chose form a selection of all the Arithmetic mathematical operations for doing my calculations   \\
                 \hline
                 \textbf{Acceptance}& 
                
                 - I know I am done When the operations related to the selected opened is displayed and available for selection 
                
                \\
                 \hline
                 \textbf{Estimate} &  0.4  point  \\
                 \hline
                 \textbf{Constrains}& The display cannot start from 0 ie Input 01 = "1" / Must not accept characters  \\
                 \hline
                \end{tabular}
            \hfill\break\\\\
        
         \section{US3 - Subtract two Numbers }
                \begin{tabular}{ |p{4cm}|p{10cm}| }
                 \hline
                 \multicolumn{2}{|c|}{\textbf{US3 - Subtract two Numbers} } \\
                 \hline
                 \textbf {Story ID}& US3  \\
                 \hline
                 \textbf{Priority} & HIGH \\
                 \hline
                 \textbf{Description}   &  As an user of calculator,I want the calculators be able to subtract the two numbers so that I can see the difference between them or use them as apart of a further calculation\\
                 \hline
                 \textbf{Acceptance}& 
                
                 -I know I am done when, The difference of two numbers ie 2-1=1
                
                \\
                 \hline
                 \textbf{Estimate} &  0.2  point  \\
                 \hline
                 \textbf{Constrains}& The calculator should display accurately \textbf{2} Decimal places when showing the results \\
                 \hline
                \end{tabular}
            
   \newpage
         \section{US4 - Add two Numbers }
                \begin{tabular}{ |p{4cm}|p{10cm}| }
                 \hline
                 \multicolumn{2}{|c|}{\textbf{US4 - Add two Numbers } } \\
                 \hline
                 \textbf {Story ID}& US4  \\
                 \hline
                 \textbf{Priority} & MEDIUM \\
                 \hline
                 \textbf{Description}   &  s an user of calculator, I want the calculators be able to add the two numbers so that I can see the Sum of the two numbers or use them as apart of a further calculations\\
                 \hline
                 \textbf{Acceptance}& 
                
                 I know I am done when, The addition of two numbers ie 2+1=3
                
                \\
                 \hline
                 \textbf{Estimate} &  0.2  point  \\
                 \hline
                 \textbf{Constrains}& The calculator should display accurately \textbf{2} Decimal places when showing the results \\
                 \hline
                \end{tabular}
            \hfill\break\\
            
            
            \section{US5 - Divide two Numbers }
                \begin{tabular}{ |p{4cm}|p{10cm}| }
                 \hline
                 \multicolumn{2}{|c|}{\textbf{US6 - Divide two Numbers} } \\
                 \hline
                 \textbf {Story ID}& US5  \\
                 \hline
                 \textbf{Priority} & MEDIUM \\
                 \hline
                 \textbf{Description}   & As an user of calculator, I want the  calculator to be able to divide my two entered Numbers and show the result so that I can see their division and/maybe use them as apart of a further calculation \\
                 \hline
                 \textbf{Acceptance}& 
                
                 I know I am done when,The Division of two numbers is 2/2=1
                
                \\
                 \hline
                 \textbf{Estimate} &  0.2  point  \\
                 \hline
                 \textbf{Constrains}& The calculator should display accurately \textbf{2} Decimal places when showing the results \\
                 \hline
                \end{tabular}
            \hfill\break\\
            
            
              \section{US6 - Multiply Two Numbers  }
                \begin{tabular}{ |p{4cm}|p{10cm}| }
                 \hline
                 \multicolumn{2}{|c|}{\textbf{US6 - Multiply two Numbers } } \\
                 \hline
                 \textbf {Story ID}& US6  \\
                 \hline
                 \textbf{Priority} & MEDIUM \\
                 \hline
                 \textbf{Description}   & As an user of calculator, I want the  calculator to be able to multiply my two entered Numbers and show the result so that I can see their multiplication and/maybe use them as apart of a further calculation \\
                 \hline
                 \textbf{Acceptance}&
                
                 I know I am done when,The Multiplication of two numbers is ie 2*2=4
                
                \\
                 \hline
                 \textbf{Estimate} &  0.2  point  \\
                 \hline
                 \textbf{Constrains}& The calculator should display accurately \textbf{2} Decimal places when showing the results \\
                 \hline
                \end{tabular}
            \hfill\break\\
            
            
              \section{US7 - Calculate the Universal Parabolic Constant}
                \begin{tabular}{ |p{4cm}|p{10cm}| }
                 \hline
                 \multicolumn{2}{|c|}{\textbf{US7 - Calculate the Universal Parabolic Constant} } \\
                 \hline
                 \textbf {Story ID}& US7  \\
                 \hline
                 \textbf{Priority} & HIGH \\
                 \hline
                 \textbf{Description}   & As an user of calculator, I to able to utilize the Universal Parabolic Constant as a constant in my calculations/functions \\
                 \hline
                 \textbf{Acceptance}& 
                
                 I know I am done when, I press the "P" constant and the output is approximately "2.29558714939 \\
                 \hline
                 \textbf{Estimate} &  1  point  \\
                 \hline
                 \textbf{Constrains}& The calculator should accurately display  \textbf{10} Decimal places when showing the results \\
                 \hline
                \end{tabular}
            \hfill\break\\
            
             \section{US8 - Calculate the Average distance from Arc Extrema}
                \begin{tabular}{ |p{4cm}|p{10cm}| }
                 \hline
                 \multicolumn{2}{|c|}{\textbf{US8 -Calculate the Average distance from Arc Extrema} } \\
                 \hline
                 \textbf {Story ID}& US8  \\
                 \hline
                 \textbf{Priority} & HIGH \\
                 \hline
                 \textbf{Description}   & As an user of calculator, I want to calculate the Arc distance utilizing the Universal parabolic constant from my entered number point and display the average distance to the arc extrema  \\
                 \hline
                 \textbf{Acceptance}& 
                
                 I know I am done when,  When the result output is within the range of the the actual average distance. \\
                 \hline
                 \textbf{Estimate} &  1  point  \\
                 \hline
                 \textbf{Constrains}& The result output should be displayed in "cm" value  \\
                 \hline
                \end{tabular}
            \hfill\break\\
            
            
            
               \section{US9 - Calculate the Approximate Surface Area}
                \begin{tabular}{ |p{4cm}|p{10cm}| }
                 \hline
                 \multicolumn{2}{|c|}{\textbf{US9 - Calculate the Approximate Surface Area} } \\
                 \hline
                 \textbf {Story ID}& US9  \\
                 \hline
                 \textbf{Priority} & LOW \\
                 \hline
                 \textbf{Description}   & As an user of calculator, I want to be able to calculate the average surface area within a simulated arc utilizing the Universal parabolic constant and display the average aproximate surface area covered by the arc. \\
                 \hline
                 \textbf{Acceptance}& 
                
                 I know I am done when,  When the result output is within the range of the the actual average area. \\
                 \hline
                 \textbf{Estimate} &  0.5  point  \\
                 \hline
                 \textbf{Constrains}& The result output should be displayed in "cm" value  \\
                 \hline
                \end{tabular}
            \hfill\break\\
             
        
        
            
        \end{quote}
     
        
      
        

        
      


    
 \newpage

\chapter{Backwards Tractability Matrix}


Tractability is useful instrument in SRS documentation which allows development teams to analyse documentations and trace problems to their initial inceptions, as one of the requirements for this project's documentation process, enlisted user stories are back traced to their initial sources to allow relateable contex to their eventual implementations,

    
\begin{quote}
\centering 
\hfill

\begin{tabular}{|p{3cm}|p{3cm}|p{3cm}|p{3cm}|p{3cm}|}

\hline
& \textbf{Interviewee} & \textbf{Online Sources} & \textbf{Personal Hypothesis }\\
\hline

US\#1&&&\checkmark\\
\hline
US\#2 &&&\checkmark\\
\hline
US\#3&&&\checkmark\\
\hline

US\#4&&&\checkmark\\
\hline

US\#5&&&\checkmark\\
\hline

US\#6&&&\checkmark\\
\hline

US\#7&\checkmark&&\\
\hline

US\#8&\checkmark&&\\
\hline

US\#9&\checkmark&\checkmark&\\
\hline


\end{tabular}
\end{quote}
\chapter{References}
\begin{quote}
    

\begin{enumerate}
\item  {\it Universal parabolic constant\/} www.revolvy.com/page/Universal-parabolic-constant?cr=1,2005).
\item Nick Babich, The Art of the User Interview, {medium.springboard.com/the-art-of-the-user-interview-cf40d1ca62e8 , Oct 9, 2017}
\item  {\it Universal parabolic constant Wolfram Analysis\\}  
http://mathworld.wolfram.com/UniversalParabolicConstant.html.
\end{enumerate}


        

\end{quote}



\end{document}



















